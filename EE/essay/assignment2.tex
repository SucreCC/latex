%%%%%%%%%%%%%%%%%%%%%%%%%%%%%%%%%%%%%%%%%
% Short Sectioned Assignment
% LaTeX Template
% Version 1.0 (5/5/12)
%
% This template has been downloaded from:
% http://www.LaTeXTemplates.com
%
% Original author:
% Frits Wenneker (http://www.howtotex.com)
%
% License:
% CC BY-NC-SA 3.0 (http://creativecommons.org/licenses/by-nc-sa/3.0/)
%
%%%%%%%%%%%%%%%%%%%%%%%%%%%%%%%%%%%%%%%%%

%----------------------------------------------------------------------------------------
%	PACKAGES AND OTHER DOCUMENT CONFIGURATIONS
%----------------------------------------------------------------------------------------

\documentclass[paper=a4, fontsize=11pt]{scrartcl} % A4 paper and 11pt font size

\usepackage[T1]{fontenc} % Use 8-bit encoding that has 256 glyphs
\usepackage{fourier} % Use the Adobe Utopia font for the document - comment this line to return to the LaTeX default
\usepackage[english]{babel} % English language/hyphenation
\usepackage{amsmath,amsfonts,amsthm} % Math packages

\usepackage{lipsum} % Used for inserting dummy 'Lorem ipsum' text into the template

\usepackage{sectsty} % Allows customizing section commands
\allsectionsfont{\centering \normalfont\scshape} % Make all sections centered, the default font and small caps

\usepackage{fancyhdr} % Custom headers and footers

% use for graph
\usepackage{graphicx} 
\usepackage{subfigure}
\usepackage{caption}


\pagestyle{fancyplain} % Makes all pages in the document conform to the custom headers and footers
\fancyhead{} % No page header - if you want one, create it in the same way as the footers below
\fancyfoot[L]{} % Empty left footer
\fancyfoot[C]{} % Empty center footer
\fancyfoot[R]{\thepage} % Page numbering for right footer
\renewcommand{\headrulewidth}{0pt} % Remove header underlines
\renewcommand{\footrulewidth}{0pt} % Remove footer underlines
\setlength{\headheight}{13.6pt} % Customize the height of the header

\numberwithin{equation}{section} % Number equations within sections (i.e. 1.1, 1.2, 2.1, 2.2 instead of 1, 2, 3, 4)
\numberwithin{figure}{section} % Number figures within sections (i.e. 1.1, 1.2, 2.1, 2.2 instead of 1, 2, 3, 4)
\numberwithin{table}{section} % Number tables within sections (i.e. 1.1, 1.2, 2.1, 2.2 instead of 1, 2, 3, 4)

\setlength\parindent{0pt} % Removes all indentation from paragraphs - comment this line for an assignment with lots of text

%----------------------------------------------------------------------------------------
%	TITLE SECTION
%----------------------------------------------------------------------------------------

\newcommand{\horrule}[1]{\rule{\linewidth}{#1}} % Create horizontal rule command with 1 argument of height

\title{	
\normalfont \normalsize 
\textsc{University College cork} \\ [25pt] % Your university, school and/or department name(s)
\horrule{0.5pt} \\[0.4cm] % Thin top horizontal rule
\huge The ethical issues in the use of AI in healthcare \\ % The assignment title
\horrule{2pt} \\[0.5cm] % Thick bottom horizontal rule
}

\author{Kai Deng} % Your name

\date{\normalsize\today} % Today's date or a custom date


\begin{document}
\maketitle % Print the title

%----------------------------------------------------------------------------------------
%	PROBLEM 1
%----------------------------------------------------------------------------------------

\section{Introduction}






Based on the improvement of certain theories and the improvement of computer computing power,
AI has achieved rapid development in the 21st century. It raises huge expectaions, has attracted 
significant investment expecialy in Medical and healthcare area (table 1). 


\begin{figure}[htbp]
    \centering
    \begin{minipage}[t]{0.48\linewidth}
        \includegraphics[width=\linewidth]{./data/investegatement.png}
        \caption{Annual private investment in AI}
        \label{fig:investment}
    \end{minipage}\hfill
    \begin{minipage}[t]{0.48\linewidth}
        \includegraphics[width=\linewidth]{./data/influence.png}
        \caption{Views on AI's impact on society}
        \label{fig:views_ai_impact}
    \end{minipage}
\end{figure}





So far, advocates of healthcare (HCAI)
have promised the thechnology will improve the accuracy of screening and diagnosis, increase the availability of create
in remote regions and free up physicain's time so that they can engage more with patients \cite{frostPublicViewsEthical2022}.
Meanwhile, questions around the potential exacerbation of health disparities due to modeling biases have raised notable ethical
concerns regarding the use of this technology in healthcare \cite{onianiAdoptingExpandingEthical2023}. 
Here we are going to disscuess, how did this issues arise and what shold we do to solve it.



% \begin{figure}[h]
%     \centering
%     \includegraphics[width=0.9\textwidth]{./data/investegatement.png}
%     \caption{This is the title}
%     \label{fig:my_picture}
% \end{figure}



% \begin{figure}[h]
%     \centering
%     \includegraphics[width=0.9\textwidth]{./data/influence.png}
%     \caption{This is the title}
%     \label{fig:my_picture}
% \end{figure}







% 问题是怎么产生的
% 监管
% 大众
% 人工智能的制作者


% 处理这个问题的原则
% 人工智能遵循原则和


% 结论






%----------------------------------------------------------------------------------------

\bibliographystyle{plain} % This specifies the style of the bibliography
\bibliography{/Users/dengkai/workspace/papers/latex/config/ref} % This should match the name of your .bib file without the extension


\end{document}

