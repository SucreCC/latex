%%%%%%%%%%%%%%%%%%%%%%%%%%%%%%%%%%%%%%%%%
% Short Sectioned Assignment
% LaTeX Template
% Version 1.0 (5/5/12)
%
% This template has been downloaded from:
% http://www.LaTeXTemplates.com
%
% Original author:
% Frits Wenneker (http://www.howtotex.com)
%
% License:
% CC BY-NC-SA 3.0 (http://creativecommons.org/licenses/by-nc-sa/3.0/)
%
%%%%%%%%%%%%%%%%%%%%%%%%%%%%%%%%%%%%%%%%%

%----------------------------------------------------------------------------------------
%	PACKAGES AND OTHER DOCUMENT CONFIGURATIONS
%----------------------------------------------------------------------------------------

\documentclass[paper=a4, fontsize=11pt]{scrartcl} % A4 paper and 11pt font size

\usepackage[T1]{fontenc} % Use 8-bit encoding that has 256 glyphs
\usepackage{fourier} % Use the Adobe Utopia font for the document - comment this line to return to the LaTeX default
\usepackage[english]{babel} % English language/hyphenation
\usepackage{amsmath,amsfonts,amsthm} % Math packages

\usepackage{lipsum} % Used for inserting dummy 'Lorem ipsum' text into the template

\usepackage{sectsty} % Allows customizing section commands
\allsectionsfont{\centering \normalfont\scshape} % Make all sections centered, the default font and small caps

\usepackage{fancyhdr} % Custom headers and footers

% use for graph
\usepackage{graphicx} 
\usepackage{subfigure}
\usepackage{caption}
\usepackage{float} 

\pagestyle{fancyplain} % Makes all pages in the document conform to the custom headers and footers
\fancyhead{} % No page header - if you want one, create it in the same way as the footers below
\fancyfoot[L]{} % Empty left footer
\fancyfoot[C]{} % Empty center footer
\fancyfoot[R]{\thepage} % Page numbering for right footer
\renewcommand{\headrulewidth}{0pt} % Remove header underlines
\renewcommand{\footrulewidth}{0pt} % Remove footer underlines
\setlength{\headheight}{13.6pt} % Customize the height of the header

\numberwithin{equation}{section} % Number equations within sections (i.e. 1.1, 1.2, 2.1, 2.2 instead of 1, 2, 3, 4)
\numberwithin{figure}{section} % Number figures within sections (i.e. 1.1, 1.2, 2.1, 2.2 instead of 1, 2, 3, 4)
\numberwithin{table}{section} % Number tables within sections (i.e. 1.1, 1.2, 2.1, 2.2 instead of 1, 2, 3, 4)

\setlength\parindent{0pt} % Removes all indentation from paragraphs - comment this line for an assignment with lots of text

%----------------------------------------------------------------------------------------
%	TITLE SECTION
%----------------------------------------------------------------------------------------

\newcommand{\horrule}[1]{\rule{\linewidth}{#1}} % Create horizontal rule command with 1 argument of height

\title{	
\normalfont \normalsize 
\textsc{University College cork} \\ [25pt] % Your university, school and/or department name(s)
\horrule{0.5pt} \\[0.4cm] % Thin top horizontal rule
\huge The ethical issues in the use of AI in healthcare \\ % The assignment title
\horrule{2pt} \\[0.5cm] % Thick bottom horizontal rule
}

\author{Kai Deng} % Your name

\date{\normalsize\today} % Today's date or a custom date


\begin{document}
\maketitle % Print the title

%----------------------------------------------------------------------------------------
%	PROBLEM 1
%----------------------------------------------------------------------------------------

\section{Introduction}

In the 21st century, advancements in theory and computational power have rapidly propelled 
artificial intelligence (AI), especially in healthcare, drawing significant investments (see Figures 1.1 and 1.2). 
Proponents believe AI can enhance diagnostic accuracy, extend care to remote areas, and save doctors' 
time for more patient interaction \cite{frostPublicViewsEthical2022}. However, AI's potential to worsen 
health disparities due to biases has sparked ethical concerns about privacy, data ownership, biased system risks, 
and lack of human oversight \cite{onianiAdoptingExpandingEthical2023, katiraiEthicsAdvancingArtificial2023}. 
This text will explore these issues' in public and docter angle and give some suggestings.

\begin{figure}[H]
    \centering
    \begin{minipage}[t]{0.48\linewidth}
        \includegraphics[width=\linewidth]{./data/investment_by_type.png}
        \caption{Annual investment in AI by type}
        \label{fig:investment}
    \end{minipage}\hfill
    \begin{minipage}[t]{0.48\linewidth}
        \includegraphics[width=\linewidth]{./data/investment_by_area.png}
        \caption{Annual investment in AI by area}
        \label{fig:views_ai_impact}
    \end{minipage}
\end{figure}



\section{Reasons for ethical issues in public}
% Public may be both benificiaries of new HCAI technologies and the greatest sufferers of AI-related harms. The result highlight that there are still noticeable concerns about 
% implementing HCAI in diagnostics and treatment recommendations for patients eith both acute and chronic illnesses, even if these tools are used as a recommendation system under the 
% physician experience and wisdom. Individuals may still not be ready to accept and use HCAI \cite{esmaeilzadehPatientsPerceptionsHuman2021}. Patients and publics are important voices In
% developing effective and ethical AI governance, but engaging patients and pbulics meanigfully in research about ethical HCAI is challenging. Most people have no first hand experience with HCAI,
% and some are unfamiliar with the concept of AI in general, Pulics may have limited understaing of how HCAI may be implemented and limited knowledge about the potential wrongs and harms that coudld
% arise from implementing HCAI \cite{frostPublicViewsEthical2022}, which means The public, as an actor, has yet to become more aware of privacy, data ownership, and human rights.

The public stands to gain from new Healthcare AI (HCAI) technologies but also faces the highest risk of 
AI-related issues. Concerns persist around HCAI's role in diagnosing and suggesting treatments 
for both acute and chronic conditions, despite physician oversight. Many are hesitant to embrace HCAI \cite{esmaeilzadehPatientsPerceptionsHuman2021}. 
Engaging patients and the wider public in ethical HCAI discussions is crucial yet challenging. 
Most lack direct HCAI experience or even a basic understanding of AI. Further more the following figure 2.1 and figure 2.2 indicate a prevailing 
global sentiment of unease regarding the societal impact of artificial intelligence over the next two decades. 
Particularly, the older population and individuals with lower economic status are more inclined to believe that 
AI will bring more harm than benefit, highlighting widespread caution about AI's future role.
Finally, public also concerns about the leakage of private health information, Since it will affects the personal life, including bullying, high insurance premium,
and loss of job due to the medical history \cite{Thapa2021PrecisionHealthData}.


% Their limited insight into HCAI's 
% application and the potential for harm highlights the need for greater awareness around privacy, data rights, 
% and human rights \cite{frostPublicViewsEthical2022}, emphasizing the importance of informed public participation in AI governance.




\begin{figure}[htbp]
    \centering
    \begin{minipage}[t]{0.48\linewidth}
        \includegraphics[width=\linewidth]{./data/influence_by_ages.png}
        \caption{Views on AI's impact on society by ages}
        \label{fig:investment}
    \end{minipage}\hfill
    \begin{minipage}[t]{0.48\linewidth}
        \includegraphics[width=\linewidth]{./data/influence_by_demographic_group.png}
        \caption{Views on AI's impact on society by gender and wealth}
        \label{fig:views_ai_impact}
    \end{minipage}
\end{figure}


% 医生
\section{Reasons for ethical issues in doctors}
% For doctors, doctors are in a very awkward position in the entire system. Healthcare artificial intelligence is patient-centered during its implementation.

% On the one hand, artificial intelligence systems are said to be the solution for many highly skilled medical tasks 
% where machines have the potential to surpass human capabilities, such as identifying normal and abnormal chest X-rays \cite{iniestaHumanRoleGuarantee2023}.
% The transformative power of data technology has brought worries and fears to doctors. Machines and 
% doctors seem to be placed on a double-edged balance to determine who will go and who will stay. 
% They are afraid that they will be replaced by machines.
% Or would disempowerment of clinicians, resulting in the development stage of HCAI, 
% clinicians do not fully trust Artificial intelligen system or developers, ultimately affecting the performance of HCAI.

Artificial intelligence is heralded as a breakthrough for complex medical tasks, such as accurately interpreting chest X-rays, 
potentially outperforming humans. However, this innovation stirs fear among doctors about being replaced or marginalized, 
highlighting a lack of trust in AI and its developers, which could hamper AI's effectiveness in healthcare. 

In Additionally, a national survey in Turkey, involving 167 emergency medicine specialists, indicated a majority downplay ethical concerns about data storage 
and reuse. 61.68\% trust in anonymity to protect privacy, and 70.66\% believe AI systems are unbiased \cite{ahunPerceptionsConcernsEmergency2023}, 
which shows Understanding the ethics of artificial intelligence is also crucial for docter.



\section{suggestings}
(1) By educating the public and medical staff on artificial intelligence and related ethical issues, this will help 
to improve their awareness of privacy and human rights in artificial intelligence.
\\(2) By openly acknowledging the value of medical staff and respecting the decisions of clinicians, they realize that 
artificial intelligence is not a competitive relationship but a cooperative relationship, and the output of artificial 
intelligence is communicated to patients in a way that doctors can understand.

\section{Conclusion}



Anxiety towards AI in healthcare stems from a lack of understanding and uncertainty about responsibility boundaries. 
Addressing this requires enhancing AI ethics and education for all involved, clarifying role boundaries to mitigate 
ethical issuess arising from AI use. 

The application potential of artificial intelligence in the medical field is huge, but in the era of artificial intelligence, 
we should not forget that people are not just composed of data. Even when talking about personalized medicine, we should keep 
asking ourselves: "Where are the people in AI-based personalized medicine?"
Personhood is a profound concept related to phenomenal consciousness, intention, and free will. If clinical recommendations 
automatically deployed by AI are directly integrated, this will preemptively prevent clinicians from developing their own clinical 
judgment capabilities, which we humans do when developing AI tools.
Artificial intelligence should be ensured to safeguard our health and well-being, and especially our dignity as human beings.


% \begin{figure}[h]
%     \centering
%     \includegraphics[width=0.9\textwidth]{./data/investegatement.png}
%     \caption{This is the title}
%     \label{fig:my_picture}
% \end{figure}



% \begin{figure}[h]
%     \centering
%     \includegraphics[width=0.9\textwidth]{./data/influence.png}
%     \caption{This is the title}
%     \label{fig:my_picture}
% \end{figure}
















%----------------------------------------------------------------------------------------

\bibliographystyle{unsrt} % This specifies the style of the bibliography
\bibliography{/Users/dengkai/workspace/papers/latex/config/ref} % This should match the name of your .bib file without the extension


\end{document}

