%%%%%%%%%%%%%%%%%%%%%%%%%%%%%%%%%%%%%%%%%
% Short Sectioned Assignment
% LaTeX Template
% Version 1.0 (5/5/12)
%
% This template has been downloaded from:
% http://www.LaTeXTemplates.com
%
% Original author:
% Frits Wenneker (http://www.howtotex.com)
%
% License:
% CC BY-NC-SA 3.0 (http://creativecommons.org/licenses/by-nc-sa/3.0/)
%
%%%%%%%%%%%%%%%%%%%%%%%%%%%%%%%%%%%%%%%%%

%----------------------------------------------------------------------------------------
%	PACKAGES AND OTHER DOCUMENT CONFIGURATIONS
%----------------------------------------------------------------------------------------

\documentclass[paper=a4, fontsize=11pt]{scrartcl} % A4 paper and 11pt font size

\usepackage[T1]{fontenc} % Use 8-bit encoding that has 256 glyphs
\usepackage{fourier} % Use the Adobe Utopia font for the document - comment this line to return to the LaTeX default
\usepackage[english]{babel} % English language/hyphenation
\usepackage{amsmath,amsfonts,amsthm} % Math packages

\usepackage{lipsum} % Used for inserting dummy 'Lorem ipsum' text into the template

\usepackage{sectsty} % Allows customizing section commands
\allsectionsfont{\centering \normalfont\scshape} % Make all sections centered, the default font and small caps

\usepackage{fancyhdr} % Custom headers and footers

% use for graph
\usepackage{graphicx} 
\usepackage{subfigure}
\usepackage{caption}
\usepackage{float} 

\pagestyle{fancyplain} % Makes all pages in the document conform to the custom headers and footers
\fancyhead{} % No page header - if you want one, create it in the same way as the footers below
\fancyfoot[L]{} % Empty left footer
\fancyfoot[C]{} % Empty center footer
\fancyfoot[R]{\thepage} % Page numbering for right footer
\renewcommand{\headrulewidth}{0pt} % Remove header underlines
\renewcommand{\footrulewidth}{0pt} % Remove footer underlines
\setlength{\headheight}{13.6pt} % Customize the height of the header

\numberwithin{equation}{section} % Number equations within sections (i.e. 1.1, 1.2, 2.1, 2.2 instead of 1, 2, 3, 4)
\numberwithin{figure}{section} % Number figures within sections (i.e. 1.1, 1.2, 2.1, 2.2 instead of 1, 2, 3, 4)
\numberwithin{table}{section} % Number tables within sections (i.e. 1.1, 1.2, 2.1, 2.2 instead of 1, 2, 3, 4)

\setlength\parindent{0pt} % Removes all indentation from paragraphs - comment this line for an assignment with lots of text

%----------------------------------------------------------------------------------------
%	TITLE SECTION
%----------------------------------------------------------------------------------------

\newcommand{\horrule}[1]{\rule{\linewidth}{#1}} % Create horizontal rule command with 1 argument of height

\title{	
\normalfont \normalsize 
\textsc{University College cork} \\ [25pt] % Your university, school and/or department name(s)
\horrule{0.5pt} \\[0.4cm] % Thin top horizontal rule
\huge Modeling Biological Neural Networks \\ % The assignment title
\horrule{2pt} \\[0.5cm] % Thick bottom horizontal rule
}

\author{Kai Deng} % Your name

\date{\normalsize\today} % Today's date or a custom date

% 参考

% https://zhuanlan.zhihu.com/p/66585918
% https://www.jiqizhixin.com/articles/spiking-neurons

\setlength{\abovecaptionskip}{10pt} % 设置图注上方的间隔为5pt
\setlength{\belowcaptionskip}{10pt} % 设置图注下方的间隔为-10pt

\begin{document}
\maketitle % Print the title



\section{introduction}
For a comprehensive understanding of modeling biological neural networks, 
I recommend reading several seminal works. "Real-time computing without stable
 states: a new framework for neural computation based on perturbations" by Maass 
 et al. (2002) provides a framework for neural computation that departs from 
 traditional stable state computations. Additionally, Markram et al. (1997) 
 have contributed foundational knowledge on synaptic efficacy and its regulation. 
 To delve into contemporary neuromorphic computing systems, Mead (1990) and Meier (2015) 
 offer significant insights. Finally, for advancements in phase-change memory devices and 
 their application to neuromorphic computing, you might consider exploring the works of Tuma et al. 
 (2016) and Burr et al. (2017). These papers will offer a broad and detailed spectrum of 
 knowledge on the subject.

 For additional papers on modeling biological neural networks, you may look into 
 "Neuromorphic computing using non-volatile memory" by Burr et al. (2017), 
 which explores the application of non-volatile memory technology in neural network models, 
 and Davies et al. (2018) discuss "Loihi", a neuromorphic manycore processor with on-chip learning 
 capabilities. For a tutorial on brain-inspired computing using phase-change memory devices,
see the work by Sebastian et al. (2018). These papers are at the forefront of combining neuroscience 
with advanced computational technologies.

Certainly! Here are some recent significant papers related to the modeling of biological neural networks which you may find valuable:

1. A paper on "Reliable interpretability of biology-inspired deep neural networks" delves into how interpretations from neural networks can be influenced by the initialization of weights and the structure of the network itself. The study emphasizes the need for repeated network training to ensure robustness and reliability of interpretations (Nature Systems Biology and Applications).

2. The article "Deep learning incorporating biologically inspired neural dynamics and in-memory computing" discusses learning spatio-temporal patterns and the use of phase-change memristors, highlighting advancements in neuromorphic computing and its potential applications in biological neural network modeling (Nature Machine Intelligence).

3. In "Biologically informed deep neural network for prostate cancer discovery," researchers have presented a framework for outcome prediction and hypothesis generation in prostate cancer, demonstrating how a deep neural network can incorporate biological pathways to provide insights that could potentially be translated clinically (Nature).

4. For a more theoretical perspective, the arXiv paper "Learning biological neuronal networks with artificial neural networks: neural oscillations" provides insights into the parameters and simulations of spiking neural networks (SNNs), focusing on how they can be used to understand neural oscillations, which are crucial in various cognitive computations.

These papers offer a blend of practical applications, theoretical understanding, and the latest methodologies in the field. They reflect the current trends in harnessing deep learning and computational models to gain insights into biological neural networks and their functions.












% 当人们谈及机器学习,深度学习,Geoffrey Hinton、Yann LeCun 和 Yoshua Bengio。 因为这三位科学家在深度学习和神经网络领域做出了突出贡献,特别是在反向传播算法、卷积神经网络、循环神经网络等方面,他们被认为是现代神经网络理论的关键推动者。
% 今天我们讲点不一样的,我们讲讲,Warren McCulloch 和 Walter Pitts







\subsection{Background Context}
\subsubsection{Anatomy of a Neuron}
\paragraph{Biological neural networks}
 are complex networks composed of nerve cells (neurons) in organisms (especially human brains and animal brains) and the connections between them. This network is the basis of the biological nervous system and is responsible for processing and transmitting information, allowing organisms to perceive the environment, make decisions, control movement and perform various complex behaviors.

\begin{figure}[H]
    \centering
    \caption{Human Neural Networks}
    \includegraphics[width=0.9\textwidth]{./data/neural-netwrok.jpg}
    \footnotesize{Source:Cajal’s drawing of a Purkinje cell. Notice the complexity and number of dendrite branches..}
    \label{fig:my_picture}
\end{figure}

\paragraph{Neurons}
receive signals from other neurons through synapses located – not exclusively – on their dendritic tree, which is a complex, branching, sprawling structure. If you are wondering what the king of dendritic complexity is, that would be the Purkinje cell, which may receive up to 100,000 other connections. Dendrites are studded with dendritic spines – little bumps where other neurons make contact with the dendrite.

\vspace{10pt}
Signals from the dendrites propagate to and converge at the soma – the cell’s body where nucleus and other typical cell organelles live.

\vspace{10pt}
Coming off the soma is the axon hillock which turns into the axon. The axon meets other neurons at synapses. It allows a neuron to communicate rapidly over long distances without losing signal integrity. To allow signals to travel rapidly, the axon is myelinated – it is covered with interspersed insulators which allows the neuron’s signal to jump between insulated sections. To allow the signal to maintain integrity, the neuron signal in the axon is ‘all-or-nothing’ – it is a rather bit-like impulse, which we will discuss next.


\begin{figure}[H]
    \centering
    \caption{Anatomy of the neuron}
    \includegraphics[width=0.9\textwidth]{./data/neural.jpg}
    \footnotesize{Source:“Neurons and glial cells” by OpenStax College, Biology CC BY-NC-SA 3.0 License.}
    \label{fig:my_picture}
\end{figure}


\subsubsection{Physiology of a Neuron}
\paragraph{The}

 second thing to appreciate about neurons is their specialized physiology — that is the cellular functions of neurons. The most striking feature of neural cellular function is the action potential. This is the mechanism which allows neurons to transmit information reliably over long distances without the transmission attenuating.

\vspace{10pt}
It is important to remember that neurons bathe in an extracellular solution of mostly water, salts and proteins. The forces caused by the movement of salts into and out of the cell and the different concentrations of these salts is the physical basis of the neuron’s remarkable behavior. There are sodium-potassium pumps which move sodium out of the cell and potassium in, so that the concentration of sodium outside the cell is higher than inside and the concentration of potassium outside the cell is lower then inside.

\vspace{10pt}
An action potential is a discrete event in which the membrane potential rapidly rises (depolarizes) and then falls (polarizes). This discrete event is all-or-nothing, meaning that if an action potential occurs at one part of the neurons membrane, it will also occur in the neighboring part, and so on until it reaches the axon terminal. Action potentials do not tend to travel backwards, because, once a section of the membrane has fired an action potential, electrochemical-forces hyper-polarize the region of membrane while the channels which were previously open close and become inactive for some duration of time.

\vspace{10pt}
The action potential is the result of different species of ions traveling across the cell membrane through channels and the activation and inactivation of those channels on different time scales. A stereotypical action potential occurs as follows:



\begin{itemize}
    \item \textbf{Equilibrium:} The neuron’s equilibrium membrane potential is near \(-70\) mV — roughly the Nernst Equilibrium of \( E_{K^+} \approx -75 \). At equilibrium, the net current is \(0\) — inward and outward currents are balanced.
    \item \textbf{Depolarization:} Incoming excitatory signals depolarize the membrane. Quick-to-respond voltage gated \( Na^+ \) channels are activated, and \( Na^+ \) rushes in, pushing the membrane potential higher. Slower-to-respond \( K^+ \) channels open, and \( K^+ \) rushes out, pushing the membrane potential lower.
    \item \textbf{Amplification:} If the neuron becomes more stimulated or is stimulated rapidly, many more \( Na^+ \) channels are activated than \( K^+ \) channels. This causes a feedback loop where the influx of \( Na^+ \) causes more \( Na^+ \) channels to activate.
    \item \textbf{Repolarization:} Eventually the membrane potential is near the Nernst Equilibrium of \( Na^+ \) as the sodium channels are maximally open. The slower \( K^+ \) channels catch up to \( Na^+ \), which repolarizes the membrane potential. Meanwhile, the \( Na^+ \) channels become inactive.
    \item \textbf{Hyper-polarization:} \( K^+ \) channels are open while \( Na^+ \) channels are inactive, causing the membrane potential to dip below its typical equilibrium point, near the \( K^+ \) Nernst equilibrium.
    \item \textbf{Refractory Period:} The \( Na^+ \) channels, take a while to become deinactivated, meaning after an action potential, they remain incapable of opening again for a period of time. The period in which most \( Na^+ \) channels are called the absolute refractory period (the neuron cannot spike no matter the strength of the stimulus) while the period in which many \( Na^+ \) channels are inactivated is called the relative refractory period (the neuron can spike given a sufficiently strong stimulus).
\end{itemize}













\subsection{Significance of Modeling}
Modeling serves as a pivotal tool in neuroscience, providing insight into the functionality of the brain and the intricate mechanisms of neural processes. It offers a window into the otherwise inaccessible workings of neural communications.

\subsection{Historical Overview}
The endeavor to model neural networks is not new. Groundbreaking models such as the Hodgkin-Huxley and FitzHugh-Nagumo models have been instrumental in advancing our understanding of neural behavior and have set the stage for modern developments in neural modeling.

\subsection{Challenges and Limitations}
Accurately modeling biological neural networks presents numerous challenges. These include the complexity of neuronal dynamics, the nonlinear nature of neural responses, and the vast interconnectivity within the neural network.

\subsection{Recent Advances}
Recent advancements in computational power and mathematical methodologies have allowed for the creation of more sophisticated and detailed models of neural networks, pushing the boundaries of what was previously possible.

\subsection{Applications}
The application of neural network models is vast, ranging from the exploration of neurological disorders, such as epilepsy and Alzheimer’s disease, to the investigation of cognitive processes.

\subsection{Objectives of Your Work}
The objectives of the presented work are to \ldots (here, you would specify the goals of your own research or modeling approach).






%----------------------------------------------------------------------------------------

\bibliographystyle{unsrt} % This specifies the style of the bibliography
\bibliography{/Users/dengkai/workspace/papers/latex/config/ref} % This should match the name of your .bib file without the extension
\end{document}

